\documentclass[11pt,a4paper,twoside]{report}
  \usepackage{a4wide}
  \usepackage{epsfig}
  \usepackage{amsmath}
  \usepackage{tabu}
  \usepackage{amsfonts}
  \usepackage{latexsym}
  \usepackage[utf8]{inputenc}
  \usepackage{listings}
  \usepackage{color}
  \usepackage{titlesec}    
  \usepackage{enumitem}
  \usepackage[catalan]{babel}
  \usepackage{newunicodechar}
  \usepackage{graphicx}
  \usepackage{subcaption}
  \usepackage{float}
  \usepackage[numbered,framed]{matlab-prettifier}
  \usepackage{xcolor}
  \usepackage{pgf, tikz}
  \usetikzlibrary{arrows, automata, positioning, datavisualization, datavisualization.formats.functions}
  
\setcounter{tocdepth}{4}
\setcounter{secnumdepth}{4}
  
\newunicodechar{Ŀ}{\L.}
\newunicodechar{ŀ}{\l.}


% \titleformat{\chapter}
%   {\normalfont\LARGE\bfseries}{\thechapter}{1em}{}
% \titlespacing*{\chapter}{0pt}{3.5ex plus 1ex minus .2ex}{2.3ex plus .2ex}

\definecolor{dkgreen}{rgb}{0,0.6,0}
\definecolor{gray}{rgb}{0.5,0.5,0.5}
\definecolor{mauve}{rgb}{0.58,0,0.82}

\lstset{frame=tb,
language=Matlab,
aboveskip=3mm,
belowskip=3mm,
showstringspaces=false,
columns=flexible,
basicstyle={\small\ttfamily},
numbers=none,
numberstyle=\tiny\color{gray},
keywordstyle=\color{blue},
commentstyle=\color{dkgreen},
stringstyle=\color{mauve},
breaklines=true,
breakatwhitespace=true,
tabsize=3,
extendedchars=true,
literate={á}{{\'a}}1 {à}{{\`a}}1 {ã}{{\~a}}1 {é}{{\'e}}1 {è}{{\`e}}1 {í}{{\'i}}1 {ï}{{\"i}}1 {ó}{{\'o}}1 {ò}{{\`o}}1 {ú}{{\'u}}1 {ü}{{\"u}}1 {ç}{{\c{c}}}1
			{Á}{{\'A}}1 {À}{{\`A}}1 {Ã}{{\~A}}1 {É}{{\'E}}1 {È}{{\`E}}1 {Í}{{\'I}}1 {Ï}{{\"I}}1 {Ó}{{\'O}}1 {Ò}{{\`O}}1 {Ú}{{\'U}}1 {Ü}{{\"U}}1 {Ç}{{\c{C}}}1
}


\usepackage{hyperref}
\hypersetup{
  colorlinks=false, %set true if you want colored links
  linktoc=all,     %set to all if you want both sections and subsections linked
  linkcolor=blue,  %choose some color if you want links to stand out
}


\newcommand\double[3][10]{%Passantli A i B genera quatre vertexs virtuals A-B-s, A-B-e (per resepresentar una aresta) i B-A-s, B-A-e (per representar l'altre aresta)
  \draw (#2)
    edge [bend left=#1,draw=none]
    coordinate[at start](#2-#3-s)
    coordinate[at end](#2-#3-e)
    (#3)
    edge [bend right=#1,draw=none]
    coordinate[at start](#3-#2-e)
    coordinate[at end](#3-#2-s)
    (#3);
}

\setlength{\footskip}{50pt}
\setlength{\parindent}{0cm} \setlength{\oddsidemargin}{-0.5cm} \setlength{\evensidemargin}{-0.5cm}
\setlength{\textwidth}{17cm} \setlength{\textheight}{23cm} \setlength{\topmargin}{-1.5cm} \addtolength{\parskip}{2ex}
\setlength{\headsep}{1.5cm}

% "define" Scala
\lstdefinelanguage{scala}{morekeywords={class,object,trait,extends,with,new,if,while,for,def,val,var,this},
otherkeywords={->,=>},
sensitive=true,
morecomment=[l]{//},
morecomment=[s]{/*}{*/},
morestring=[b]"}

% Default settings for code listings
\lstset{frame=tb,language=scala,aboveskip=3mm,belowskip=3mm,showstringspaces=false,columns=flexible,basicstyle={\small\ttfamily}}

\renewcommand{\contentsname}{Continguts}
%\renewcommand{\chaptername}{Pr\`actica}
\setcounter{chapter}{0}

\begin{document}

\title{Pràctica Scala}
\author{Enric Rodriguez, Marc Cané}
\date{29 de novembre de 2018}
\maketitle

\tableofcontents


\chapter{Exemple de titol}

Quan parlem de matrius disperses ens referim a matrius de grans dimensions en la qual la majoria d'elements son zero. Direm que una matriu és dispersa, quan hi hagi benefici en aplicar els mètodes propis d'aquestes. 

Per identificar si una matriu és dispersa, podem usar el següent:

\qquad Una matriu $n \times n$ serà dispersa si el número de coeficients no nuls es $n^{\gamma+1}$, on $\gamma < 1$.

En funció del problema, decidim el valor del paràmetre $\gamma$. Aquí hi ha els valors típics de $\gamma$:
\begin{itemize}
\item $\gamma=0.2$ per problemes d'anàlisi de sistemes elèctrics de generació i de transport d'energia.
\item $\gamma=0.5$ per matrius en bandes associades a problemes d'anàlisi d'estructures.
\end {itemize}

\chapter{Titol 2}

\section{Per Coordenades}

És la primera aproximació que podríem pensar i és bastant intuïtiva. Per cada element no nul guardem una tupla amb el valor i les seves coordenades: $(a_{i j}, i, j)$. 

A la realitat però, aquest mètode d'emmagatzemar les dades és poc eficient quan hem de fer operacions amb les matrius.

\section{Per files}  
	
També conegut com a \textit{Compressed Sparse Rows (CSR)}, \textit{Compressed Row Storage (CRS)}, o format \textit{Yale}. És el mètode més estès.

Consisteix en guardar els elements ordenats per files, guardar la columna on es troben, i la posició del primer element de cada fila en el vector de valors.
Així ens quedaran tres vectors:
\begin{itemize}
	\item \textbf{valors:} de mida $n_z$, conté tots els valors diferents.
	\item \textbf{columnes:} també de mida $n_z$, conté la columna on es troba cada un dels elements anteriors.
	\item \textbf{iniFiles:} de mida $m+1$, conté la posició on comença cada fila en els vectors valors i columnes, sent $m$ el nombre de files de la matriu. 
\end{itemize}

\subsection{Exemple}

Si es canvien files per columnes, dona la implementació per columnes, o també anomenada \textit{Compressed Sparse Columns (CSC)}.

\subsection{Implementació del mètode CSR}

Hem implementat un script Matlab amb una classe \texttt{CSRSparseMatrix} que guardi les dades necessàries. Aquestes les tenim en ``l'atribut"\texttt{ Matrix} dins del bloc \texttt{properties} (línia 10 del codi següent). Aquestes dades consisteixen en el següent:	
\begin{itemize}
\item  \texttt{Matrix.nColumns}: número de columnes de la matriu, necessari per recrear les files posteriorment.
\item  \texttt{Matrix.values}: vector valors comentat anteriorment, amb els valors no nuls de la matriu.
\item  \texttt{Matrix.columns}: vector de columnes, amb la columna corresponent a cada valor amb el mateix índex.
\item  \texttt{Matrix.beginningRow}: vector amb els índex comença cada fila en el vector de valors i de columnes.
\end{itemize}


\newpage
\section {Codi}

\paragraph*{Exemple codi Scala:}\mbox{}\\

\lstinputlisting[language=Scala]{report_src/exemple.scala}

%\lstinputlisting[language=Scala]{../src/Main.scala}
 
\end{document}
